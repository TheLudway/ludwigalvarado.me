%
%Academic CV LaTeX Template
% Author: Dubasi Pavan Kumar
% LinkedIn: https://www.linkedin.com/in/im-pavankumar/
% License: MIT
%
% For errors, suggestions, or improvements, please contact:
% Email: pavankumard.pg19.ma@nitp.ac.in
%



\documentclass[a4paper,11pt]{article}

% Package imports
\usepackage{latexsym}
\usepackage{xcolor}
\usepackage{float}
\usepackage{ragged2e}
\usepackage[empty]{fullpage}
\usepackage{wrapfig}
\usepackage{lipsum}
\usepackage{tabularx}
\usepackage{titlesec}
\usepackage{geometry}
\usepackage{marvosym}
\usepackage{verbatim}
\usepackage{enumitem}
\usepackage{fancyhdr}
\usepackage{multicol}
\usepackage{graphicx}
\usepackage{cfr-lm}
\usepackage[T1]{fontenc}
\usepackage{fontawesome5}

% Color definitions
\definecolor{darkblue}{RGB}{0,0,139}

% Page layout
\setlength{\multicolsep}{0pt}
\pagestyle{fancy}
\fancyhf{} % clear all header and footer fields
\fancyfoot{}
\renewcommand{\headrulewidth}{0pt}
\renewcommand{\footrulewidth}{0pt}
\geometry{left=1.4cm, top=0.8cm, right=1.2cm, bottom=1cm}
\setlength{\footskip}{5pt} % Addressing fancyhdr warning

% Hyperlink setup (moved after fancyhdr to address warning)
\usepackage[hidelinks]{hyperref}
\hypersetup{
    colorlinks=true,
    linkcolor=darkblue,
    filecolor=darkblue,
    urlcolor=darkblue,
}

% Custom box settings
\usepackage[most]{tcolorbox}
\tcbset{
    frame code={},
    center title,
    left=0pt,
    right=0pt,
    top=0pt,
    bottom=0pt,
    colback=gray!20,
    colframe=white,
    width=\dimexpr\textwidth\relax,
    enlarge left by=-2mm,
    boxsep=4pt,
    arc=0pt,outer arc=0pt,
}

% URL style
\urlstyle{same}

% Text alignment
\raggedright
\setlength{\tabcolsep}{0in}

% Section formatting
\titleformat{\section}{
  \vspace{-4pt}\scshape\raggedright\large
}{}{0em}{}[\color{black}\titlerule \vspace{-7pt}]

% Custom commands
\newcommand{\resumeItem}[2]{
  \item{
    \textbf{#1}{\hspace{0.5mm}#2 \vspace{-0.5mm}}
  }
}

\newcommand{\resumePOR}[3]{
\vspace{0.5mm}\item
    \begin{tabular*}{0.97\textwidth}[t]{l@{\extracolsep{\fill}}r}
        \textbf{#1}\hspace{0.3mm}#2 & \textit{\small{#3}}
    \end{tabular*}
    \vspace{-2mm}
}

\newcommand{\resumeSubheading}[4]{
\vspace{0.5mm}\item
    \begin{tabular*}{0.98\textwidth}[t]{l@{\extracolsep{\fill}}r}
        \textbf{#1} & \textit{\footnotesize{#4}} \\
        \textit{\footnotesize{#3}} &  \footnotesize{#2}\\
    \end{tabular*}
    \vspace{-2.4mm}
}

\newcommand{\resumeProject}[4]{
\vspace{0.5mm}\item
    \begin{tabular*}{0.98\textwidth}[t]{l@{\extracolsep{\fill}}r}
        \textbf{#1} & \textit{\footnotesize{#3}} \\
        \footnotesize{\textit{#2}} & \footnotesize{#4}
    \end{tabular*}
    \vspace{-2.4mm}
}

\newcommand{\resumeSubItem}[2]{\resumeItem{#1}{#2}\vspace{-4pt}}

\renewcommand{\labelitemi}{$\vcenter{\hbox{\tiny$\bullet$}}$}
\renewcommand{\labelitemii}{$\vcenter{\hbox{\tiny$\circ$}}$}

\newcommand{\resumeSubHeadingListStart}{\begin{itemize}[leftmargin=*,labelsep=1mm]}
\newcommand{\resumeHeadingSkillStart}{\begin{itemize}[leftmargin=*,itemsep=1.7mm, rightmargin=2ex]}
\newcommand{\resumeItemListStart}{\begin{itemize}[leftmargin=*,labelsep=1mm,itemsep=0.5mm]}

\newcommand{\resumeSubHeadingListEnd}{\end{itemize}\vspace{2mm}}
\newcommand{\resumeHeadingSkillEnd}{\end{itemize}\vspace{-2mm}}
\newcommand{\resumeItemListEnd}{\end{itemize}\vspace{-2mm}}
\newcommand{\cvsection}[1]{%
\vspace{2mm}
\begin{tcolorbox}
    \textbf{\large #1}
\end{tcolorbox}
    \vspace{-4mm}
}

\newcolumntype{L}{>{\raggedright\arraybackslash}X}%
\newcolumntype{R}{>{\raggedleft\arraybackslash}X}%
\newcolumntype{C}{>{\centering\arraybackslash}X}%

% Commands for icon sizing and positioning
\newcommand{\socialicon}[1]{\raisebox{-0.05em}{\resizebox{!}{1em}{#1}}}
\newcommand{\ieeeicon}[1]{\raisebox{-0.3em}{\resizebox{!}{1.3em}{#1}}}

% Font options
\newcommand{\headerfonti}{\fontfamily{phv}\selectfont} % Helvetica-like (similar to Arial/Calibri)
\newcommand{\headerfontii}{\fontfamily{ptm}\selectfont} % Times-like (similar to Times New Roman)
\newcommand{\headerfontiii}{\fontfamily{ppl}\selectfont} % Palatino (elegant serif)
\newcommand{\headerfontiv}{\fontfamily{pbk}\selectfont} % Bookman (readable serif)
\newcommand{\headerfontv}{\fontfamily{pag}\selectfont} % Avant Garde-like (similar to Trebuchet MS)
\newcommand{\headerfontvi}{\fontfamily{cmss}\selectfont} % Computer Modern Sans Serif
\newcommand{\headerfontvii}{\fontfamily{qhv}\selectfont} % Quasi-Helvetica (another Arial/Calibri alternative)
\newcommand{\headerfontviii}{\fontfamily{qpl}\selectfont} % Quasi-Palatino (another elegant serif option)
\newcommand{\headerfontix}{\fontfamily{qtm}\selectfont} % Quasi-Times (another Times New Roman alternative)
\newcommand{\headerfontx}{\fontfamily{bch}\selectfont} % Charter (clean serif font)

\begin{document}
\headerfontiii
\begin{center}
    {\Huge\textbf{Ludwig Alvarado Becerra}}\\
    {\Large Estudiante de Ingeniería de Sistemas y Ciencia de Datos}
\end{center}
\vspace{-6mm}

\begin{center}
    \small{
    +57-320-9992989 | \href{mailto:me@ludwigalvarado.me}{me@ludwigalvarado.me} |
    \href{https://ludwigalvarado.me/}{ludwigalvarado.me} |
    \socialicon{\faLinkedin} \href{https://www.linkedin.com/in/theludway/}{Ludway} |
    \socialicon{\faGithub} \href{https://github.com/theLudway/}{TheLudway}
    }
\end{center}

\section{\textbf{Resumen}}
\small{
Estudiante de Ingeniería de Sistemas y Ciencia de Datos con sólidas habilidades en programación, algoritmos y análisis de datos. Colaborador activo en comunidades de software libre y de código abierto, con liderazgo demostrado en organizaciones estudiantiles e iniciativas de mapeo. Experiencia en coordinación de proyectos, programación competitiva y aprendizaje automático aplicado. Apasionado por usar la tecnología y los datos para el impacto social, la innovación y la construcción de comunidad.
}

\section{\textbf{Experiencia Profesional}}
\resumeSubHeadingListStart

\resumeSubheading
    {Asistente de Profesor}{Bogotá, Colombia}
    {Universidad de Bogotá Jorge Tadeo Lozano}{Feb. 2025 -- Presente}
    \resumeItemListStart
      \item Guié a los estudiantes en las asignaturas de Estructuras de Datos y Aplicaciones Móviles, brindando apoyo académico y retroalimentación.
      \item Diseñé ejercicios prácticos y promoví el entrenamiento en programación competitiva.
    \resumeItemListEnd

\resumeSubheading
    {Miembro Fundador y Representante Legal Suplente}{Bogotá, Colombia}
    {Asociación de Cartografía Colaborativa de Colombia}{Nov. 2024 -- Presente}
    \resumeItemListStart
      \item Automatización de la configuración de estaciones de trabajo Linux mediante scripts en Bash, herramientas GIS y gestión de usuarios.
      \item Coordinación de iniciativas de cartografía colaborativa junto a ONG, estudiantes y comunidades.
    \resumeItemListEnd

\resumeSubHeadingListEnd

%-------------------- Experiencia Académica --------------------
\section{\textbf{Experiencia Académica}}
\resumeSubHeadingListStart

\resumeSubheading
    {Fundador y Líder}{Bogotá, Colombia}
    {Segmentation Fault – Grupo de Programación Competitiva}{Nov. 2024 -- Presente}
    \resumeItemListStart
      \item Creé el primer grupo universitario de programación competitiva, impulsando la participación estudiantil en esta disciplina.
      \item Organicé entrenamientos y concursos semanales, brindando mentoría en algoritmos y estructuras de datos.
    \resumeItemListEnd

\resumeSubheading
    {Representante Estudiantil}{Bogotá, Colombia}
    {Facultad de Tecnología e Industrias, Utadeo}{Mar. 2024 -- Presente}
    \resumeItemListStart
      \item Promoví mejoras curriculares alineadas con las necesidades de la industria.
      \item Facilitación de la comunicación entre estudiantes, profesores y administración.
    \resumeItemListEnd

\resumeSubHeadingListEnd



\section{\textbf{Formación Académica}}
\resumeSubHeadingListStart

\resumeSubheading
    {Doble Titulación: Ingeniería de Sistemas y Ciencia de Datos}{Bogotá, Colombia}
    {Universidad de Bogotá Jorge Tadeo Lozano}{Graduación estimada: Nov. 2026}
    \resumeItemListStart
      \item Promedio: 4.4/5.0
      \item Cursos relevantes: Estructuras de Datos, Computación en la Nube, Sistemas Dinámicos.
      \item Distinciones: Premios en Feria de Ingeniería (2.º y 3.º lugar, 2023–2024).
    \resumeItemListEnd

\resumeSubheading
    {Bachiller Técnico – Mecatrónica Industrial}{Bogotá, Colombia}
    {Instituto Técnico Industrial F.J. de Caldas}{Nov. 2021}
    \resumeItemListStart
      \item Mejor Proyecto de Grado: Diseño y construcción de una máquina moldeadora de alimentos utilizando Arduino y neumática.
    \resumeItemListEnd

\resumeSubHeadingListEnd





\section{\textbf{Proyectos}}
\resumeSubHeadingListStart

\resumeProject
  {Maratón de Programación Utadeo}{Infraestructura para Concurso Local de Programación}{En curso}{\href{https://github.com/SegmentationFaultUtadeo/PropuestaMaratonUtadeo}{\faGithub}}
\resumeItemListStart
      \item Construcción y despliegue de un entorno de concursos de programación competitiva utilizando infraestructura Dockerizada.
\resumeItemListEnd

\resumeProject
  {FilmCast}{Aplicación Web y Móvil}{Ago. 2024 -- Nov. 2024}{\href{https://github.com/alaixgg/FilmCast}{\faGithub}}
    \resumeItemListStart
      \item Desarrollo de un algoritmo de aprendizaje automático para recomendación de actores; implementación de backend REST API; despliegue con Docker y servidor local.
    \resumeItemListEnd

\resumeProject
  {Dao}{Aplicación Móvil}{Mar. 2024 -- Jun. 2024}{\href{https://github.com/SpanishSyntax/Dao}{\faGithub}}
    \resumeItemListStart
      \item Diseño de una aplicación para apoyar el aprendizaje del chino (niveles HSK), aplicando principios de diseño centrado en el usuario.
    \resumeItemListEnd

\resumeSubHeadingListEnd


\section{\textbf{Habilidades}}
\small{
\textbf{Programación:} Python, C++, Bash, SQL, \LaTeX \\
\textbf{DevOps y Herramientas:} Git, GitHub Actions, Docker, Administración de Sistemas Linux, CI/CD, Azure DevOps (básico) \\
\textbf{Otras:} QGIS, JOSM, Emacs (org-mode)
}


\section{\textbf{Reconocimientos y Premios}}
\resumeSubHeadingListStart
\item Premios Feria de Ingeniería – Utadeo (2.º y 3.º lugar, 2023–2024).
\item Mejor Proyecto de Grado – Mecatrónica (2021).
\resumeSubHeadingListEnd


\section{\textbf{Voluntariado}}
\resumeSubHeadingListStart

\resumeProject
  {Fundador, Capítulo TadeoMappers}{YouthMappers}{Jul. 2024 -- Presente}{ }
\resumeItemListStart
  \item Organización de talleres de mapeo humanitario con HOT y UNGRD para respuesta a desastres.
  \item Capacitación en JOSM y QGIS para más de 20 estudiantes, contribuyendo con datos de biodiversidad e infraestructura en OpenStreetMap.
\resumeItemListEnd

\resumeProject
  {Fundador, GNUTADEO}{Colectivo de Software Libre}{Feb. 2024 -- Presente}{ }
\resumeItemListStart
  \item Promoción del uso de GNU/Linux en la universidad, incluyendo eventos de instalación y capacitaciones.
  \item Coordinación de FLISoL 2024 con más de 300 asistentes, fomentando el compromiso con la comunidad de software libre.
\resumeItemListEnd

\resumeProject
  {Coordinador, FLISoL 2024--2026}{Festival Latinoamericano de Instalación de Software Libre}{Abr. 2024 -- Presente}{ }
\resumeItemListStart
  \item Lideré la organización de la edición universitaria de FLISoL, el mayor festival de software libre en Latinoamérica.
  \item Gestión logística de talleres, charlas y eventos de instalación con más de 850 participantes.
  \item Coordinación de voluntarios y colaboración con comunidades locales de software libre para garantizar el éxito del evento.
\resumeItemListEnd

\resumeSubHeadingListEnd

\section{\textbf{Información Adicional}}
\small{
\textbf{Idiomas:} Español (nativo), Inglés (B2), Chino (HSK3). \\
\textbf{Intereses:} Programación competitiva, comunidades de software libre, tecnología geoespacial, IA para el bien social.
}

\end{document}
